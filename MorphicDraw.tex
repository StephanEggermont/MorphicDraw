\documentclass[10pt]{article}   	% use "amsart" instead of "article" for AMSLaTeX format
\usepackage{geometry}                		% See geometry.pdf to learn the layout options. There are lots.
\usepackage[english]{babel}
\geometry{a4paper}                   		% ... or a4paper or a5paper or ... 
%\geometry{landscape}                		% Activate for for rotated page geometry
%\usepackage[parfill]{parskip}    		% Activate to begin paragraphs with an empty line rather than an indent
\usepackage{graphicx}				% Use pdf, png, jpg, or epsß with pdflatex; use eps in DVI mode
\usepackage{fontspec}
\usepackage{multicol}
\setmainfont{Arial}
					
\linespread{1.1}			% TeX will automatically convert eps --> pdf in pdflatex		
\usepackage{amssymb}
\usepackage{eurosym}
\usepackage{colortbl}
\title{MorphicDraw}
\author{Stephan J.C. Eggermont, Sensus}
\begin{document}
\setlength{\parindent}{0pt}
\maketitle
\begin{quote}
\em
MorphicDraw is a drawing application demonstrating some of the power of Morphic.
Morphic is a powerful graphics environment, used in Self, Squeak, Cuis and Pharo.
In an iterative and incremental process we'll build up an application that supports
drawing connected figures.
\end{quote} 

\section{A Morphic Application}
\begin{figure}[htb]
\begin{center}
\includegraphics[width=200pt]{SimpleMorphicDrawWindow.pdf}
\caption{A first iteration of the main window of MorphicDraw}
\label{1stIteration}
\end{center}
\end{figure}
The first iteration (Figure \ref{1stIteration})  shows an application window 
with a toolbar and a drawing area. In the drawing area there are 
three graphical shapes, one of which is selected. A context menu
for the selected shape shows options to delete it and to connect it.

\subsection{An application with a window}
Add a class that represents the application. It has instance variables for 
the different parts. 

\begin{verbatim}
Object subclass: #MorphicDraw
    instanceVariableNames: 'window tools dock'
    classVariableNames: ''
    category: 'MorphicDraw-Model'
\end{verbatim}

Creating a window in Morphic is simple. Open a workspace and DoIt
\begin{verbatim}
StandardWindow new openInWorld 
\end{verbatim}
This creates a window and opens it on the screen. It already has default
behaviour for closing and resizing, and a default title. MorphicDraw
needs to change the title and default window size.

\begin{verbatim}
MorphicDraw>>createWindow
    window := StandardWindow new
        setLabel: 'MorphicDraw';
        extent: 400@400;
        yourself.
\end{verbatim}

StandardWindow is part of PolyMorph. PolyMorph makes the 
Window responsible for adding predefined user interface widgets
to the application Window. For that it uses the TEasilyThemed 
trait. It adds a lot (163 in my current image) of convenience methods.

In Morphic, a toolbar in a window has buttons on it. This iteration of
MorphicDraw uses two buttons to be able to create two different 
graphical shapes.

\begin{verbatim}
MorphicDraw>>createNewCardButton
    ^ window
        newButtonFor: self
        getState: nil
        action: #newCard
        arguments: nil
        getEnabled: nil
        labelForm: MDIcons default cardIcon
        help: 'New Card' translated
\end{verbatim}
The help text is shown when hovering the mouse over the button.
The button is always enabled, and sends the \#newCard message 
without any arguments to self when it is pressed. It has no 
state-dependent behaviour or shape. The icon for the button
is provided by MDIcons default cardIcon.

The button for the other shape is similar:
\begin{verbatim}
MorphicDraw>>createNewRectangleButton
    ^ window
        newButtonFor: self
        getState: nil
        action: #newRectangle
        arguments: nil
        getEnabled: nil
        labelForm: MDIcons default rectangleIcon
        help: 'New Rectangle' translated
\end{verbatim}

At the class side add a method to open the application

\begin{verbatim}
MorphicDraw>>open
    ^self new open
\end{verbatim}

\section{Shapes and PasteUpMorph}

\section{Toolbar}
\section{Connecting}



\section{Selection and resizing}



\end{document}  